\documentclass{article}
\usepackage{multirow}
\usepackage[letterpaper, margin=0.5in]{geometry}
\usepackage{enumitem}
\usepackage{changepage}
\usepackage[hidelinks, breaklinks=true]{hyperref}
\usepackage{natbib}
\usepackage{bibentry}

\bibliographystyle{plain}

\pagenumbering{gobble}

\begin{document}

\nobibliography{pubs}

\begin{center}
    \textbf{\huge{Eleftheria Anastasia Beres}}

    \vspace{10pt}

    \begin{tabular}{c c c}
        552 N 68th St, Seattle, WA 98103 & 817-653-5991 & \href{mailto:eberes@cs.washington.edu}{eberes@cs.washington.edu}
    \end{tabular}

    \vspace{5pt}

    \href{https://elliberes.me}{https://elliberes.me}
    
\end{center}

\section*{Education}
\hrule
\vspace{10pt}

\begin{itemize}[leftmargin=1.5in]
    \item[2024–] \textbf{University of Washington}, Seattle, WA\\
    Ph.D. in Computer Science and Engineering\\
    \textit{In progress}
    \item[2021–2024] \textbf{Northwestern University}, Evanston, IL\\
    B.S. in Computer Science\\
    Summa Cum Laude; GPA: 3.99/4.00; Major GPA 4.00/4.00
\end{itemize}

\section*{Research Interests}
\hrule
\vspace{10pt}

Executable and algorithmic approaches to modeling biological systems; reproducibility, replicability, and reuse in computational, systems, and synthetic biology; computing education for disciplinary scientists

\section*{Research Experience}
\hrule
\vspace{10pt}

\begin{itemize}[leftmargin=1.5in]
    \item[Sep 2024–Present] \textbf{Programming Languages and Software Engineering (PLSE) Group\\and Learning, Computing, Imagination (LCI) Lab}\\
    Paul G. Allen School of Computer Science \& Engineering, Seattle, WA\\
    \textit{Research Assistant}\\
    \textit{Advisors: Michael D. Ernst, Amy J. Ko, R. Benjamin Shapiro}
    \begin{itemize}
        \item Working on improving the reproducibility of Jupyter Notebooks for scientific work in the JupyterLab interactive development environment.
        \item Studying software engineering practices among disciplinary scientists specifically in the context of reproducibility, replicability, reuse, and correctness in computational science.
    \end{itemize}
    \item[Dec 2021–Jun 2024] \textbf{Leonard Lab}\\
    Northwestern University Center for Synthetic Biology, Evanston, Il\\
    \textit{Undergraduate Researcher}\\
    \textit{Advisors: Joshua Leonard}
    \begin{itemize}
        \item Designed and implemented Python package for easy-to-use, rapid, batch flow cytometry data analysis for synthetic biologists.
        \item Experimentally characterized mammalian genetic circuit components for computer-aided design of mammalian genetic circuits.
        \item Characterized and analyzed synthetic transcription factors in mammalian cells.
    \end{itemize}
    \item[Sep 2023–Jun2024] \textbf{Xenobot Group}\\
     Northwestern University Center for Robotics and Biosystems, Evanston, IL\\
    \textit{Undergraduate Researcher}\\
    \textit{Advisor: Sam Kriegman}
    \begin{itemize}
        \item Studied the impacts of simulated growth on rigid-body virtual creatures.
        \item Used genetic algorithms to evolutionarily optimize rigid-body robotics in a physics simulator.
    \end{itemize}
\end{itemize}

\section*{Poster Presentations}
\hrule
\vspace{10pt}

\begin{itemize}[leftmargin=1.5in]
    % \item[2023] \textbf{Beres E}, et al. PyFlowBAT: An Open-Source Software Package for Performing High-Throughput Batch Analysis of Flow Cytometry Data. Poster presented at: EBRC Annual Meeting; 2023 Jun 5-6; Evanston, IL.
    \item[2023] \bibentry{pyflowbatebrc}
\end{itemize}

\section*{Teaching Experience}
\hrule
\vspace{10pt}

\begin{itemize}[leftmargin=1.5in]
    \item[Summer 2024] Peer Mentor\footnote{Northwestern did not officially have undergraduate TA positions in the McCormick School of Engineering; instead, the positions were called ``Peer Mentors.''}: \textbf{COMP\_SCI 349 Machine Learning}\\
    Department of Computer Science\\
    Northwestern University, Evanston, IL
    \item[Spring 2024] Peer Mentor: \textbf{COMP\_SCI 349 Machine Learning}\\
    Department of Computer Science\\
    Northwestern University, Evanston, IL
    \item[Winter 2024] Peer Mentor: \textbf{COMP\_SCI 396/496 Artificial Life}\\
    Department of Computer Science\\
    Northwestern University, Evanston, IL
    \item[Fall 2023] Peer Mentor: \textbf{GEN\_ENG 205-1 Engineering Analysis 1}\\
    Department of Electrical and Computer Engineering\\
    Northwestern University, Evanston, IL
    \item[Fall 2023] Peer Mentor: \textbf{BMD\_ENG 220 Introduction to Biostatistics}\\
    Department of Biomedical Engineering\\
    Northwestern University, Evanston, IL
    \item[Winter 2023] Peer Mentor: \textbf{DATA\_ENG 200 Foundations of Data Engineering}\\
    Department of Computer Science\\
    Northwestern University, Evanston, IL
    \item[Fall 2022] Peer Mentor: \textbf{BMD\_ENG 220 Introduction to Biostatistics}\\
    Department of Biomedical Engineering\\
    Northwestern University, Evanston, IL
\end{itemize}

\section*{Awards \& Honors}
\hrule
\vspace{10pt}

\begin{itemize}[leftmargin=1.5in]
    \item[2024–2027] \textbf{ARCS Scholar}\\
    ARCS Seattle Chapter, Seattle, WA
    \item[2024–2025] \textbf{Jeff Dean - Heidi Hopper Endowed Regental Fellowship}\\
    Paul G. Allen School of Computer Science \& Engineering, Seattle, WA
    \item[Jun 2024] \textbf{Outstanding Computer Science Senior}\\
    Northwestern University Department of Computer Science, Evanston, IL
    \item[Summer 2023] \textbf{Summer Undergraduate Research Fellowship}\\
    Northwestern University Department of Computer Science, Evanston, IL
    \item[Summer 2023] \textbf{Summer Undergraduate Research Grant}\\
    Northwestern University Office of Undergraduate Research, Evanston, IL
    \item[Summer 2022] \textbf{Summer Undergraduate Research Grant}\\
    Northwestern University Office of Undergraduate Research, Evanston, IL
\end{itemize}

\section*{Service \& Organizations}
\hrule
\vspace{10pt}

\begin{itemize}[leftmargin=1.5in]
    \item[Jan 2025–Present] \textbf{oSTEM at UW}: Secretary\\
    University of Washington, Seattle, WA
    \item[Jan 2025–Mar 2025] \textbf{Prospective Students Visit Days Scheduler}\\
    Paul G. Allen School of Computer Science \& Engineering, Seattle, WA
    \item[Dec 2024–Jan 2025] \textbf{Ph.D. Program Application Reader}\\
    Paul G. Allen School of Computer Science \& Engineering, Seattle, WA
    \item[Sep 2024–Dec 2024] \textbf{Pre-Application Mentor Service (PAMS)}: PAMS Mentor\\
    Paul G. Allen School of Computer Science \& Engineering, Seattle, WA
    \item[Sep 2022–Sep 2023] \textbf{NU Biomedical Engineering Society}: Secretary\\
    Northwestern University Department of Biomedical Engineering, Evanston, IL
    \item[Sep 2021–Sep 2022] \textbf{NU Biomedical Engineering Society}: Public Relations Chair\\
    Northwestern University Department of Biomedical Engineering, Evanston, IL
\end{itemize}

\section*{Relevant Coursework}
\hrule
\vspace{10pt}

\subsection*{University of Washington}

\begin{itemize}[leftmargin=1.5in]
    \item[Spring 2025] \textbf{GENOME 569 Bioinformatics Workflows for High-Throughput Sequencing Experiments}\\
    Grade: \textit{In progress} | Skills: Learning and implementing high-throughput bioinformatics workflows for single cell RNA sequencing experiments.
    \item[Spring 2025] \textbf{CSE 587 Advanced Systems and Synthetic Biology}\\
    Grade: \textit{In progress} | Skills: Learning about recent developments in synthetic biology.
    \item[Winter 2025] \textbf{CSE 503 Software Engineering}\\
    Grade: 4.00 | Skills: Learned program analysis methods and software engineering research topics.
\end{itemize}

\subsection*{Northwestern University}

\begin{itemize}[leftmargin=1.5in]
    \item[Spring 2024] \textbf{COMP\_SCI 324 Dynamics of Programming Languages}\\
    Grade: 4.00 | Skills: Performed mathematical modeling of programming languages; created redex model of the semantics of the future annotation in MultiLisp based on ``The Semantics of Future'' by Cormac Flanagan and Matthias Felleisen.
    \item[Winter 2024] \textbf{COMP\_SCI 397 Computer Science Education}\\
    Grade: 4.00 | Skills: Learned constructionist approaches to CS education; created tool to demonstrate the similarities between spreadsheets and traditional programming.
    \item[Fall 2023] \textbf{COMP\_474 Probabilistic Graphical Models}\\
    Grade: 4.00 | Skills: Covered statistical estimation and graphical models including Naive Bayes models, Markov Networks, Factor Graphs, Gaussian Mixture Models, Expectation Maximization, Latent Dirichlet Allocation.
    \item[Winter 2023] \textbf{ES\_APPM 346 Modeling and Computation in Science and Engineering}\\
    Grade: 4.00 | Skills: Learned and implemented numerical methods for solving differential equations including stochastic differential equations.
    \item[Fall 2022] \textbf{BMD\_ENG 311 Computational Genomics}\\
    Grade: 4.00 | Skills: Learned bioinformatics methods for analyzing single cell RNA sequencing data; analyzed previously-published genomics data from the NCBI GEO database from Zika and Dengue infections to identify differentially expressed genes.
\end{itemize}

\section*{Skills}
\hrule
\vspace{10pt}

\begin{itemize}[leftmargin=1.5in]
    \item[Computing] Unix; Python; C; TypeScript and JavaScript; Java; Go; Julia.
    \item[Machine learning] Implementing neural networks from scratch; genetic algorithms; probabilistic graphical models.
    \item[Bioinformatics] Performing scRNA-seq analysis; automating bioinformatics pipelines with Bash, Python, and R; flow cytometry analysis.
    \item[Biological wetlab] Mammalian cell culture; plasmid design and prep; HEK cell transfection; flow cytometry.
\end{itemize}

\end{document}
